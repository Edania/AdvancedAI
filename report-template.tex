\documentclass[12pt,a4paper]{article}

\usepackage[a4paper, margin=2.5cm, top=2.5cm, bottom=2.0cm]{geometry}  % proper margins
\usepackage{amsmath, amssymb}  % extended math environment and symbols
\usepackage[us,24hr]{datetime} % `us' makes \today behave as usual in TeX/LaTeX
\usepackage{fancyhdr}
\usepackage[T1]{fontenc}  % handling umlauts etc in output
\usepackage{glossaries}  % abbreviations
\usepackage{graphicx}  % figures
\usepackage[utf8]{inputenc}  % handling umlauts etc in input
\usepackage{listings}  % code examples
\usepackage{palatino}  % main font
\usepackage{siunitx}  % units

\usepackage[dvipsnames]{xcolor}
\usepackage[
    unicode,
    colorlinks=true,
    urlcolor=NavyBlue,
    linkcolor=NavyBlue,
    citecolor=NavyBlue
]{hyperref}

% comment this out if you would like to remove the 'Version' timestamp in the header of page 1
\fancypagestyle{plain}{
\fancyhf{}
\rhead{\footnotesize Version {\ddmmyyyydate\today} at \currenttime}
\renewcommand{\headrulewidth}{0pt}}

% example for how to declare an abbreviation using the glossaries package
\newacronym{rmse}{RMSE}{root-mean-square error}
\newacronym{mse}{MSE}{mean squared error}
\newacronym{mcmc}{MCMC}{Markov chain Monte Carlo}

% example for how to declare non-standard units using the siunitx package
\DeclareSIUnit{\atom}{atom}

\begin{document}

% This sets the default language for the listings package.
\lstset{language=Python}

\title{
    \sffamily
    Report on Project <project-number>: \\
    <project-title>
}
\author{Bobby Tables}
\date{\today}

\maketitle

This is a course on \textit{Advanced Simulation and Machine Learning}. Although it is a physics course, it has an obvious focus on statistics, i.e., inference, optimization, parameter estimation, etc. It is therefore important that you discuss your findings and demonstrate your ability to use appropriate terminology to describe the employed statistical, analytical, and computational methodologies.\\

Please refer to the respective project documents for additional instructions.

\section{Some recommendations}

You can use this document as a starting point for your report.
It has been sensibly (from our perspective) configured and demonstrates some potentially useful packages.

\subsection{Abbreviations}

We recommend using the \texttt{glossaries} package for handling abbreviations.
It will take care of making sure an abbreviation is introduced on its first occasion in the text.
For example, the \gls{rmse} is ridiculously low but only if the \gls{rmse} is less than two elephants.

\subsection{Units}

We recommend using the \texttt{siunitx} package for handling units.
For example, the Suez Canal measures \SI{193.3}{\kilo\meter}, a value that has been vetted by the scrupulous editors of Wikipedia.
Believe it or not but the speed limit for scooters is \SI{20}{\kilo\meter\per\hour} according to \href{https://scotsman.me/post/what-s-the-law-on-riding-a-scooter-in-sweden}{this webpage}.
Sometimes you might also just want to denote a unit in its full glory, such as \si{\nano\meter\squared}.

\section{General instructions}

The following points are based on the experience in previous years.
We assume that all of you are familiar with these aspects but it is useful to keep them in mind when writing the reports.

\subsection{Units}

\begin{itemize}
\item 
    All numbers must be reported with proper units.
\item 
    Please note that if you choose to standardize the target data (e.g., the mixing energies that are originally in \si{\milli\electronvolt\per\atom} in project 2a) this unit changes upon standardization.
    This must be corrected for afterwards to obtain a \gls{rmse} that can be compared to the original data in a meaningful way.
\item 
    Check your units and target data.
    If the \gls{mse} is of the order of the variance of the target data, the model is not better than guessing.
    It is strongly recommended to use the same units as the data.
\end{itemize}

\subsection{Simulations in general}

\begin{itemize}
\item 
    Clearly state how you choose your parameters.
\end{itemize}

\subsection{\texorpdfstring{\gls{mcmc}}{MCMC} simulations}

A careful \gls{mcmc} simulation should contain the following for each class of parameters (model parameters and hyper-parameters):
\begin{itemize}
\item
    An illustration of the burn-in to motivate how much of the initial data to throw away.
\item
    How the walkers were initialized.
\item
    Some measure of the decoherence length (here the chain itself can be used but preferably an auto-correlation plot should be made).
\item
    Some motivation of a successful total sampling, e.g., a histogram (it should at least be rather 'smooth') can be used for each parameter class.
    The \href{https://corner.readthedocs.io/}{\texttt{corner} package} can be a real boon for such tasks.
\end{itemize}

\subsection{Bayesian data analysis}

\begin{itemize}
\item
    Priors must be clearly stated, motivated and optionally illustrated in a figure. 
\item
    Motivate and declare your likelihood function. What are the statistical grounds for it?
\item
    It is useful plot the posteriors.
    This can help you spot errors. A \texttt{corner} plot of the posterior is often useful to include in the report. 
\item Motivate your physical model/theory.
\end{itemize}

\subsection{Tables}

\begin{itemize}
\item
    Tables are  useful for reporting/contrasting
  results. 
\end{itemize}

\subsection{Figures}

\begin{itemize}
\item 
    Figures must be reported in the correct units and have proper axis labels.
    It is insufficient to simply write ``RMSE'', ``CV'' etc.
\item
    Please make sure that your figure are in sufficiently high resolution.
    Also keep in mind that if PDF is a vector format.
    This implies that if you generate a figure with a lot of dots (or other symbols) all these data survive even if the symbols overlap (as they typically do for example in a parity plot).
    In such a case, PDF files can become very large and unwieldy.
    Under such circumstances, a bitmap format, preferably PNG, is preferable if you make sure that the resolution is still sufficiently high.
\end{itemize}

\end{document}